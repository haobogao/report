\chapter{Linux}




\chapter{系统调用}








\chapter{reboot}


kernel\_restart


\begin{lstlisting}
void kernel_restart(char *cmd)
{
	kernel_restart_prepare(cmd);
	migrate_to_reboot_cpu();
	syscore_shutdown();
	if (!cmd)
		pr_emerg("Restarting system\n");
	else
		pr_emerg("Restarting system with command '%s'\n", cmd);
	kmsg_dump(KMSG_DUMP_RESTART);
	machine_restart(cmd);
}
\end{lstlisting}


\begin{lstlisting}
arch\arm64\kernel\process.c
void machine_restart(char *cmd)
{
	/* Disable interrupts first */
	local_irq_disable();
	smp_send_stop();

	/* Now call the architecture specific reboot code. */
	if (arm_pm_restart)
		arm_pm_restart(reboot_mode, cmd);
	else
		do_kernel_restart(cmd);

	/*
	 * Whoops - the architecture was unable to reboot.
	 */
	printk("Reboot failed -- System halted\n");
	while (1);
} 

\end{lstlisting}

在msm-poweroff.c 中的msm\_restart\_probe函数中,指定了

\begin{lstlisting}
pm_power_off = do_msm_poweroff;
arm_pm_restart = do_msm_restart;
\end{lstlisting}


do\_msm\_restart

msm\_restart\_prepare




\chapter{Linux systemcall }

\begin{lstlisting}
#define __NR_fsetxattr 7
__SYSCALL(__NR_fsetxattr, sys_fsetxattr)
#define __NR_getxattr 8
__SYSCALL(__NR_getxattr, sys_getxattr)
#define __NR_lgetxattr 9
__SYSCALL(__NR_lgetxattr, sys_lgetxattr)
#define __NR_fgetxattr 10
__SYSCALL(__NR_fgetxattr, sys_fgetxattr)
#define __NR_listxattr 11
\end{\lstlisting}
在 unistd.h 中定义了系统调用映射关系 ,然后填充 sys\_call\_table:
\begin{\lstlisting}
void * const sys_call_table[__NR_syscalls] __aligned(4096) = {
	[0 ... __NR_syscalls - 1] = sys_ni_syscall,
#include <asm/unistd.h>
};
\end{lstlisting}


