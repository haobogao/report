\chapter{Arm Architecture OverView}



\section{Exception}

\subsection{Arm V8 Exception Level}


D1.1 Exception levels

arm v8 定义了四个执行级别EL0~EL3(Exception Levels),
EL后面跟的数字 越大,执行的特权级越高。
\begin{itemize}
\item EL0 , 是非特权级。通常是用户空间的applications.

\item EL1,OS kernel 运行在这个状态。

\item EL2,可以提供virtualization of Non-secure operation。对应Hypervisor

\item EL3,可以提供 Secure state and Non-secure state 之间的转换。
\end{itemize}

上层可能不会实现上述的所有特权级别,但是EL1,EL0是必须实现的,EL2,EL3 是可以选择实现的。

当在AArch64 状态下执行的时候,发生异常可以使特权级别升高或者不变,从异常中返回可以使特权级别降低或者不变。特权级的提高
可以访问:
\begin{itemize}

\item 在当前特权级和安全状态的资源.

\item 比当前特权级低的特权级的所有资源。

\end{itemize}


所以如果一个系统实现包括了EL3,当处理器执行在EL3 状态的时候,就可以访问安全状态和非安全状态的所有资源。


\subsection {Arm V8 Secure model}


arm v8 安全模型的基本策略是:

\begin{itemize}

\item 如果一个实现包括了EL3,这个实现就包含了两种状态,secure 和 Non-secure.

EL3仅仅在安全模式。 从 Non-secure 到 secure唯一的方式是  进入EL3,从secure 到 Non-secure 的唯一方式是从EL3 返回.
如果实现了EL2 ,EL2 仅在Non-secure 状态.

\item 如果没有实现EL3,则仅有一种安全状态:
\begin{itemize}

\item 如果没有实现EL2,为 IMPLEMENTATION DEFINED。 

\item 如果实现了EL2,为 Non-secure。
\end{itemize}

\end{itemize}

图 \ref{sec_mod},展示了安全模型:

\begin{figure}
\begin{center}
\includegraphics[width=15cm]{img/ELframe}
\caption{security model}
\label{sec_mod}
\end{center}
\vspace{-0.5em}
\end{figure}


\section{Registers}
D1.6
在arm架构中,Register 被分为两类:

\begin{itemize}
\item 系统控制和状态寄存器。

\item 用来数据计算和处理的寄存器。
\end{itemize}

\subsection{System Register}


\subsection{}



