%!TEX program = xelatex
% Font Size:
%   10pt, 11pt, 12pt
% Paper Size:
%   a4paper, letterpaper, a5paper, leagalpaper, executivepaper, landscape
% Font Family:
%   roman, sans
\documentclass[11pt, a4paper, roman]{moderncv}

% Style:
%   casual, classic, oldstyle, banking
\moderncvstyle{classic}
% Color:
%   blue, orange, green, red, purple, grey
\moderncvcolor{purple}
% \nopagenumbers{}
% \definecolor{color0}{rgb}{0, 0, 0}
% \definecolor{color1}{RGB}{245, 90, 7}
% \definecolor{color2}{RGB}{39, 40, 34}

% Font specify
\usepackage[UTF8, scheme = plain, heading = false]{ctex}

% Page layout
\usepackage{geometry}
\geometry{scale = 0.75}
%\setlength{\hintscolumnwidth}{4cm}           % 如果你希望改变日期栏的宽度
\AtBeginDocument{\settowidth{\hintscolumnwidth}{2014 年 -- 2018 年}}

\AtBeginDocument{\hypersetup{pdfstartview = FitH}}

% Packages
\usepackage{metalogo}
\usepackage{amsmath}
\usepackage{amsfonts}

\providecommand{\CTeX}{\relax}
\renewcommand{\CTeX}{\ensuremath{\mathbb{C}}\TeX}
\usepackage{paralist}
\setmainfont{Ubuntu}

% Self-info
\name{高}{浩博}
\title{个人简历}
\address{紫荆山南路南三环}{郑州}
\email{goal.haobo@gmail.com}
\phone[mobile]{+86~17729734085}
% \phone[fixed]{+2~(345)~678~901}
% \phone[fax]{+3~(456)~789~012}
\homepage{www.vample.info}
\homepage{github.com/haobogao/}
% \extrainfo{附加信息 (可选项)}
% \photo[<height>][<width-of-frame>]{<file-name>}
% \photo[64pt][0.4pt]{picture}
% Motto
% \quote{}

% 显示索引号;仅用于在简历中使用了引言
%\makeatletter
%\renewcommand*{\bibliographyitemlabel}{\@biblabel{\arabic{enumiv}}}
%\makeatother

% 分类索引
%\usepackage{multibib}
%\newcites{book,misc}{{Books},{Others}}

\begin{document}
\maketitle

\section{教育背景}
\cventry{2014 年 -- 2018 年}{工学学士}{河南科技大学}{}{\textit{信息工程系/软件工程}}{核心课程:数据结构,算法设计,计算机组成原理,数字电路,操作系统原理,C,C++,C\#....}

 \section{毕业论文}
 \cvitem{题目}{\emph{基于 zigbee CC2530的多点数据采集系统}}
 \cvitem{导师}{杨继松博士}
% \cvitem{说明}{\small 论文简介}


\section{计算机技能}
%\cvdoubleitem{C/C++}{熟悉,曾在老师指导下为同级同学讲课}{Cuda C}{熟悉,曾开发基于 GPU 的高性能大数计算库}

\cvitem{C \& C like}{熟悉C 语言,有大量的C 语言开发经验。了解C++,C\#,在校期间有C++的扑克牌课程设计,毕业后有C\# 串口上位机的
开发经历}
\cvitem{Linux operate system}{熟悉ubuntu 环境的配置和搭建,如samba,ssh,vim,NFS,opengrok,crosstools,以及shadowsocks,wordpress...}
\cvitem{Script}{会使用shell script 方便自己的工作,可以看懂python}
\cvitem{Document}{喜欢 markdown,\LaTeX 这样的工具整理技术文档}
\cvitem{Git}{熟悉git 和 repo}
\cvitem{Tools}{熟悉GNU 的 gcc ,makefile,linker,了解elf 文件格式,有使用makefile+shell script 管理自己的项目.}
\cvitem{Embed} {熟悉 单片机软件开发,熟悉linux 应用编程,熟悉linux 驱动编程,熟悉uboot,了解lk,uefi}

\section{外语技能}
 \cvitemwithcomment{中文}{母语}{}
\cvitemwithcomment{英语}{熟练}{CET-4}
\cvitemwithcomment{日语}{简单对话}{年轻时爱看日漫的缘故}

\section{实践背景}

\cventry{河南科技大学恒生杯全国服务机器人大赛}{负责syn6288 语音模块, 超声波测距的实现,舵机调试}{}{}{}{大赛中我们小组完成了 智能家居机器人的研发。平台使用 stm32f407.外设有 摄像头,温湿度传感器
,雨滴检测模块,红外避障,机械臂,扫地机器人改装来的移动底盘,语音模块,超声波测距模块。
}


\cventry{linux kernel \& driver 学习}{linux 驱动}{}{}{}{本人在大三开始就对linux 技术十分感兴趣,从s3c2440 开始,投入大量的时间和精力在
自学嵌入式linux 方向的东西。 在2440上实现过 一些简单的字符设备驱动,如led,key,input,platform,了解 linux 驱动和内核的基本概念(通读过ldd3 和 深入理解linux 内核)。
}
 
\cventry{stm32 显示屏控制器}{code 优化}{}{}{}{实习期间在郑州汉威光电股份有限公司负责LED 可变情报板的研发和维护。 
\begin{compactitem}
	\item 我优化了SD卡播放表的存储方式, 修复了SD卡中不知原因的数据丢失问题。 
	\item 整合了大部分难以维护的代码,封装接口,添加注释,进行测试。
	\item  引入了linux kernel 链表 使memory 动态分配管理,替代了原来无处不在的大数组,缓解了内存紧张的问题。后续又
使用链表加面向对象的思想实现了安徽治超站插播播放表。
	\item 另外在这份工作期间尝试了 freeRTOS,和 RT-thread 等其他的RTOS 方案,也尝试了LWIP 的移植和使用。
\end{compactitem}
}

\cventry{uwb室内定位}{评估}{}{}{}{在foxconn 郑州中心参与了 uwb 室内定位的方案评估。
期间主要的工作有:
\begin{compactitem}
	\item 对芯片的选型 向硬件的同事提出建议。
	\item 参与demo 的搭建,期间写了个linux 下的app,用于把串口接收到的数据向服务器上报。
	\item 在demo 的源码下,实现了shutdown 和 sleep ,协助硬件评估功耗。
	\item 了解了一些uwb 定位的原理,和代码实现。
\end{compactitem}}

\cventry{cc2530 zigbee}{设计}{}{}{}{基于 Ti  CC2530 完成了毕业设计,和汉威光电雾区诱导系统的研发尝试。
主要实现了 一些外设和传感器的驱动,和一些应用逻辑。}

\cventry{cc2640 蓝牙}{评估}{}{}{}{在富士康郑州研发中心,参与了室内定位 蓝牙方案的评估,实现了CC2640的低功耗,以及led,key 驱动。
了解了 TiOS 的 机制。}

\cventry{高通平台}{BSP}{}{}{}{在富士康郑州研发中心,作为 BSP boot 的owner ,负责 新案子的bring up. 以及boot 阶段的debug 。 
协助 FTM 的owner 出工厂测试版本。具体工作内容设计保密协议,不便更详细的展开。}

\cventry{海思Hi3516 }{BSP}{}{}{}{Hi3516 的bring up .uboot 网卡驱动的优化,ssh 的移植。}

\end{document}  